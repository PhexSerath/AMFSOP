\documentclass[a4paper,12pt]{scrartcl}
\usepackage{graphicx}
\usepackage[utf8]{inputenc}
\usepackage[ngerman]{babel}
\usepackage[T1]{fontenc}
\usepackage{amsmath}
\usepackage{tikz}
\usepackage{tabto}
\usepackage{ulem}
\usepackage{cancel}
\usepackage{makecell}
\usepackage{color}
\pagestyle{empty}
\usepackage{pdflscape}
\usepackage{lscape}
\newcommand{\n}{\newline}
\newcommand{\bs}{\textbackslash}
\begin{document}
\begin{flushleft}
Steven Müller 581583 \\
Lucas Petersen 581277 
\end{flushleft}
\begin{large}
	\textbf{Aufgabe 12}\\
\end{large}
a)\\
$V(G)=\lbrace 1,2,3,4,5\rbrace$\\
$E(G)=\lbrace(1,2),(1,3),(1,4),(2,5),(3,5),(4,5)\rbrace$\\
$C = \lbrace a,b,c,d\rbrace$\\
$Lsg=\lbrace X_{1a},X_{2b},X_{3b},X_{4b},X_{5a}\rbrace$\\
$U_{a}, U_{b}, \neg U_{c}, \neg U_{d}$\\\\

b) Die Variablen $X_{ic}$ das i ist die ID des Knotens und c ist die ID der Farbe.\\
Umgangsprachlich:\\
"Der Knoten i hat die Färbung c."\\\\

$U_{c}$ bedeutet wird die Farbe mit der ID c genutzt oder nicht.\\\\

Die erste harte Klausel bedeutet:\\
Jeder Knoten muss eine Färbung haben\\\\

Die zweite harte Klausel bedeutet:\\
Jeder Knoten kann nur eine Färbung haben\\\\

Die dritte harte Klausel bedeutet:\\
Jeder Nachbar des Knotens darf nicht die selbe Färbung haben.\\\\

Die vierte harte Klausel bedeutet:\\
Das kein Knoten diese Farbe hat, oder das die Farbe verwendet wird.\\\\

Die fünfte harte Klausel bedeutet:\\
Das diese Farbe nicht genutzt wird.\\\\


All das sind Eigenschaften die für die min.\\ Graphenfärbung erforderlich sind und hier maximieren wir nach den nicht genutzten Farben.

\end{document}